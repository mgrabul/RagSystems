\documentclass{wseas}


\author{
AUTHORS' NAMES (Capital, 12pt Times New Roman, centered)\\
Department (12pt Times New Roman, centered)\\
University (12pt Times New Roman, centered)\\
Address (12pt Times New Roman, centered)\\
COUNTRY (Capital, 12pt Times New Roman, centered)
}

%% Enter the title of your article here     %%
\title{Title of the Paper (16pt Times New Roman, Bold, Centered)}


\begin{document}
\twocolumn[
  \begin{@twocolumnfalse}
    \maketitle
    \begin{abstract}
- This is a sample of the format of your full paper. A You need to have A4-sized pages (21 x 29 cm) pages with top and bottom margins of 2.5 cm and left and right margins of 1.8 cm. Use single space. Use double- use column format after the Key-Words. Arrange the text in two columns (8.2 cm), each separated by a gap of 1 cm. Use 11 pt size Times New Roman throughout the paper except for the headlines. For the words \textit{Abstract}, \textit{Key-Words} and \textit{References} Italics. Ensure that the text on the final page is spread so that both columns finish at the same distance from the top of the page.
    \end{abstract}

    \begin{keywords}
     - Leave one blank line after the Abstract and write your Key-Words (6 - 10 words)\\
    \end{keywords}

\begin{dates}
{\break
\color{red}
Received: May 31, 2019. Revised: May 4, 2020. Accepted: May 22, 2020. Published: May 29, 2020
\break(WSEAS will fill these dates in case of final acceptance, following strictly our data base and possible email
communication)}
\end{dates}
  \end{@twocolumnfalse}
\vspace{2ex} 
]

\section{Introduction}
As you can see for the title of the paper you must use 16pt, Centered, Bold, Times New Roman.
Leave one blank line and then type AUTHORS' NAMES (in Capital, 12pt Times New Roman,
centered), Department (in 12pt Times New Roman, centered), University (in 12pt Times New Roman,
centered), Address (in 12pt Times New Roman, centered), COUNTRY (in Capital, 12pt Times New
Roman, centered). Then you must type your e-mail address and your Web Site address (both in 12pt
Times New Roman, centered). 
The heading of each section should be printed in small, 14pt, left justified, bold, Times New Roman.
You must use numbers 1, 2, 3, … for the sections' numbering and not Latin numbering (I, II, III, …)

\section{Problem Formulation}
Please, leave two blank lines between successive sections as here.

Mathematical Equations must be numbered as follows: (1), (2), $\ldots$, (99) and not (1.1), (1.2), $\ldots$, (2.1), (2.2), $\ldots$  depending on your various Sections.

\subsection{Subsection}
When including a subsection you must use, for its heading, small letters, 12pt, left justified, bold, Times New Roman as here.  
\break
\break
\break
\break
\break


\subsubsection{Sub-subsection}
When including a sub-subsection you must use, for its heading, small letters, 11pt, left justified, bold, Times New Roman as here. 

\section{Problem Solution}
Figures and Tables should be numbered as follows: Fig.1, Fig.2, $\ldots$ etc Table 1, Table 2, $\ldots$ etc.
\break

If your paper deviates significantly from these specifications, our Publishing House may not be able to include your paper in the Proceedings. 
\break

When citing references in the text of the abstract, type the corresponding number in square brackets as shown at the end of this sentence \cite{cite1}.  
\break

The authors are required to look over and verify whether the in-text citations exist in the reference list and whether all the references mentioned in the reference list exist in the in-text citations. 
\break

Also, they need to look over if the in-text citations of the Tables, Equations and Figures are properly connected with the Tables, Equations and Figures.

\section{Conclusion}
Please, follow our instructions faithfully, otherwise you have to resubmit your full paper.  This will \pagebreak  enable us to maintain uniformity in the conference proceedings as well as in the post-conference luxurious books by Press. Thank you for your cooperation and contribution. We are looking forward to seeing you at the Conference.


\flushleft \par{\textit {Acknowledgment:}}\\
It is an optional section where the authors may write a short text on what should be acknowledged regarding their manuscript.



\begin{thebibliography}{9}
\bibitem{cite1} Author, Title of the Paper, \textit{International Journal of Science and Technology}, Vol.X, No.X, 200X, pp. XXX-XXX.
\bibitem{cite2}   Author, \textit{Title of the Book}, Publishing House, 200X.

\end{thebibliography}


\flushleft \justifying{ \textbf {Contribution of Individual Authors to the \break Creation of a Scientific Article (Ghostwriting \break  Policy)}}
\par{{\color{red}
\noindent \textbf{Please, indicate the role and the contribution of each author:}\\ 
\\
For Example:}}
\par{
\noindent 
 {John Smith, Donald Smith carried out the simulation and the optimization.  }\break
George Smith has implemented the Algorithm 1.1 and 1.2 in C++.  \break
Maria Ivanova has organized and executed the experiments of Section 4.  \break
George Nikolov was responsible for the Statistics.  \break

\noindent Please visit Contributor Roles Taxonomy (CRediT) that contains several roles:
\href{www.wseas.org/multimedia/contributor-roleinstruction.pdf}{\color{blue}{www.wseas.org/multimedia/contributor-roleinstruction.pdf}}

\flushleft \justifying{ \textbf {\hl{Alternatively, the following text will be} \linebreak {\hl {published:}}}}
\justifying{ \par { \noindent 
The authors equally contributed in the present research, at all stages from the formulation of the problem to the final findings and solution.}}

\flushleft \justifying{ \textbf {Sources of Funding for Research Presented in a Scientific Article or Scientific Article Itself}}
\par{{\color{red}
\noindent \textbf{Report potential sources of funding if there is any}}}


\flushleft \justifying{ \textbf {\hl {Alternatively, in case of no funding the following text will be published:}}}
\par { \noindent No funding was received for conducting this study.}
\break

\flushleft \justifying{ \textbf {Conflicts of Interest}}
\par { \noindent {\color{red}{\textbf{Please declare anything related to this study.}}}}

\flushleft \justifying{ \textbf{\hl {Alternatively, in case of no conflicts of interest the following text will be published:}}}}
\par  \justifying{ { \noindent The authors have no conflicts of interest to \break declare that are relevant to the content of this \break  article.}}


\flushleft \justifying{ \textbf{Creative Commons Attribution License 4.0 \break (Attribution 4.0 International , CC BY 4.0)}}

\par { \noindent This article is published under the terms of the Creative Commons Attribution License 4.0\\
\href{https://creativecommons.org/licenses/by/4.0/deed.en\_US}{\color{blue}{https://creativecommons.org/licenses/by/4.0/deed.en\break\_US}} }




\end{document}

